\graphicspath{{introduction/fig/}}

\chapter{Introduction}
\label{chap:introduction}


In the modern day, jogging is one of the most popular physical activities. People all over the world partake in this physical activity. One would think that jogging is a simple exercise and there are no major injury or health concerns. However, there are some major concerns as jogging has a high injury rate. Over the past few years, there has been an increasing interest in this subject and many studies have been conducted on how these injuries can be analyzed. Strike patterns during running has been acknowledged as potential way of identifying injuries or risk of injury. 





\section{Problem Statement}
Develop an application that can use data from a prototype device and display the foot strike patterns from this data. Do additional calculations to show the ground force applied with each step. The raw data should be processed and saved to be displayed and used in useful manners such as videos and graphs. 
\section{Objectives}
% TODO: this sucks
The objective of this project is to use a prototype wearable device that captures the pressure applied to a grid of pressure sensing resistors. The prototype has been build by a previous student and is capable of capturing the data from FSR cells and transmitting the data via BLE. An mobile application should then be developed to receive this data, process it and display or save it in useful ways.The application can then be used by medical experts to assist in medical assessments like the Gait analysis. 
\section{Scope and Limitations}

The scope of this project is to only design and develop an andriod application that displays the foot strike patterns captured by the prototype device. As the device was built by a predecessor there are some design flaws that had to be considered:
\begin{itemize}
    \item Very large device
    \item The Arduino used does not have enough ADC channels and an extra component was added taking up more space. 
    \item The battery of the device is very large and adds a large amount of weights to a device that should be wearable.
    \item The charging circuit is not in working order and the battery had to be manually charged.
  \end{itemize}

\section{Overview of the report}

Chapter 2 Lit review

Chapter 3 System design

Chapter 4 Detailed design

Chapter 5 Tests

Chapter 6 Summary conclusions


\section{Section heading}

This is some section with two table in it: Table~\ref{tbl:exemplars} and Table~\ref{tbl:abx_speaker}.

\begin{table}[!h]
    \mytable
    \caption{Performance of the unconstrained segmental Bayesian model on TIDigits1 over iterations in which the reference set is refined.}
    \begin{tabularx}{\linewidth}{@{}lCCCCC@{}}
        \toprule
        Metric     & 1 & 2 & 3 & 4 & 5 \\
        \midrule
        WER (\%)                        & $35.4$ & $23.5$ & $21.5$ & $21.2$ & $22.9$ \\
        Average cluster purity (\%)       & $86.5$ & $89.7$ & $89.2$ & $88.5$ & $86.6$ \\
        Word boundary $F$-score (\%)         & $70.6$ & $72.2$ & $71.8$ & $70.9$ & $69.4$ \\
        Clusters covering 90\% of data   & 20             & 13 & 13 & 13 & 13 \\
        \bottomrule
    \end{tabularx}
    \label{tbl:exemplars}
\end{table}


\begin{table}[!h]
    \renewcommand{\arraystretch}{1.1}
    \centering
    \caption{A table with an example of using multiple columns.}
    \begin{tabularx}{0.65\linewidth}{@{}lCCr@{}}
        \toprule
        & \multicolumn{2}{c}{Accuracy (\%)} \\
        \cmidrule(lr){2-3}
        Model    & Intermediate & Output & Bitrate\\
        \midrule
        Baseline & 27.5         & 26.4   & 116 \\
        VQ-VAE   & 26.0         & 22.1   & 190 \\
        CatVAE   & 28.7         & 24.3   & 215 \\
        \bottomrule
    \end{tabularx}
    \label{tbl:abx_speaker}
\end{table}

\newpage

This is a new page, showing what the page headings looks like, and showing how to refer to a figure like Figure~\ref{fig:cae_siamese}.

\begin{figure}[!t]
    \centering
%     \includegraphics[width=\linewidth]{cae_siamese}
    \includegraphics[width=0.918\linewidth]{cae_siamese}
    \caption[I am the short caption that appears in the list of figures, without references.]{
    (a) The cAE as used in this chapter. The encoding layer (blue) is chosen based on performance on a development set.
    (b) The cAE with symmetrical tied weights. The encoding from the middle layer (blue) is always used.
    (c) The siamese DNN. The cosine distance between aligned frames (green and red) is either minimized or maximized depending on whether the frames belong to the same (discovered) word or not.
    A cAE can be seen as a type of DNN~\cite{dahl+etal_taslp12}.
    }
    \label{fig:cae_siamese}
\end{figure}


The following is an example of an equation:
\begin{equation}
P(\vec{z} | \vec{\alpha}) = \int_{\vec{\pi}} P(\vec{z} | \vec{\pi}) \, p(\vec{\pi} | \vec{\alpha}) \, \textrm{d} \vec{\pi}
= \int_{\vec{\pi}} \prod_{k = 1}^K \pi_k^{N_k} \frac{1}{B(\vec{\alpha})} \prod_{k = 1}^K \pi_k^{\alpha_k - 1} \, \textrm{d} \vec{\pi}
\label{eq:example_equation}
\end{equation}
which you can subsequently refer to as~\eqref{eq:example_equation} or Equation~\ref{eq:example_equation}.
But make sure to consistently use the one or the other (and not mix the two ways of referring to equations).