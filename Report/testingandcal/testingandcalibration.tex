\graphicspath{{testingandcal/fig/}}

\chapter{Testing and Calibrations}
\label{chap:testing}

\section{Overview}
\label{sec:testingoverview}

Basic tests were conducted to ensure the device and application function as intended. After that, more tests were done by walking on different surfaces. By doing so, the system's capabilities could be displayed, and interesting observations could be made. An important thing to consider is that the shoe used is a running shoe, and running shoes have lifted heels to make a heel strike more comfortable. This can be seen in all tests that the heel sensors experience more pressure than the rest of the sensors. The temperature of the FSR cells was not considered, as FSR cells are not sensitive to temperature conditions, as seen in the specifications of many FSR devices. Please see figure \ref{fig:footsensordevice} in appendix \ref{appen:footweardevice} to understand the graphs' legends in this section.

\section{Calibration}
\label{sec:calibration}

The previous student did the foot sensor calibration by using a compression testing machine, the Instron 3345. According to the previous student, this machine can apply a compression force up to 40kN. The machine was set up to apply a certain amount of force and slowly release. The force was applied to a single cell and the output ADC where recorded for this cell. Due to the machine's slow release all ADC values between the interval $F_N - 1 < F_N < F_N + 1$, where $F_N$ is applied force, had been recorded and an average ADC value was calculated. The same test was done for the cells that are connected to the ADS1115. The results were recorded and plotted as seen in figure \ref{fig:calibration}
\clearpage
\begin{figure}[!htb]
    \centering
    \includegraphics[width = 0.8\linewidth]{calibration.png}
    \caption{Sensor calibration graph}
    \label{fig:calibration}
\end{figure}

\section{Applied force accuracy}
The test for this section was conducted using two 2.5kg($\approx$24.5N) weights and putting them on a single FSR cell. When testing the accuracy of the force calculations, it was found that the two cells connected externally were not accurate. Another issue was that the FSR cells had a large surface area, which made it difficult to distribute the weight evenly across a cell. If the weight is not perfectly distributed, the readings are not accurate. It is impossible to distribute the weight evenly when the foot sensor is fitted into a shoe; therefore, the sensor is not very good for accurate force measurements.

\begin{table}[!h]
    \centering
    \caption{Force readings of each FSR cell when applying 2.5kg and 5kg}
    \begin{tabular}{||c c c||} 
     \hline
     Cell & 2.5kg applied(N) & 5kg applied(N) \\ [0.5ex] 
     \hline\hline
     Arch &15.33 & 30.21 \\ 
     \hline
     Met5 &16.12 & 31.86 \\
     \hline
     Met3 &24.93& 49.51\\
     \hline
     Met1 &23.29 & 49.38 \\
     \hline
     HeelR &24.04& 48.21\\
     \hline
     HeelL &24.78& 49.78\\ 
     \hline
     Hallux &24.44& 49.37\\ 
     \hline
     Toes &24.44& 49.11\\
     \hline
    \end{tabular}
    \end{table}
\section{Bare foot}

By standing bare foot on the pressure we can see what a mid foot standing patterns should look like. This test will be used as a control as it will become apparent that the shoes influence how the device behaves.
\begin{figure}[!htb]
    \centering
    \includegraphics[width = 0.6\linewidth]{barefeet.jpg}
    \caption{Bare foot mid-stance test}
    \label{fig:calibration}
\end{figure}
\clearpage
\section{Basic strike patterns}
\label{sec:basicstrike}
\begin{figure}[!htb]
    \centering
    \includegraphics[width = 0.6\linewidth]{frontsfoot.jpg}
    \caption{Fore foot push-off with shoes}
    \label{fig:frontstrike}
\end{figure}
\begin{figure}[!htb]
    \centering
    \includegraphics[width = 0.6\linewidth]{midfoot.jpg}
    \caption{Mid-stance with shoes}
    \label{fig:midstrike}
\end{figure}
\clearpage
\begin{figure}[!htb]
    \centering
    \includegraphics[width = 0.6\linewidth]{heelstrike.jpg}
    \caption{Heel-strike with shoes}
    \label{fig:hellstrike}
\end{figure}
\clearpage
\section{Different surfaces}
\label{sec:surfaces}

The following test was conducted from a standing start and then, at a steady rate, walking across two different surfaces. When comparing the following two graphs, we can see how the grass causes a damping effect. The damping effect makes sense as the grass is soft and absorbs some force applied to the FSR cells  

\begin{figure}[!htb]
    \centering
    \includegraphics[width = 1.1\linewidth]{flatsurface.png}
    \caption{Graph that shows the force applied to each FSR cell when walking on a hard surface}
    \label{fig:hardflat}
\end{figure}

\begin{figure}[!htb]
    \centering
    \includegraphics[width = 1.1\linewidth]{grass_surface.png}
    \caption{Graph that shows the force applied to each FSR cell when walking on a grass/soft surface}
    \label{fig:softsurf}
\end{figure}
\clearpage
\section{Inclines}
\label{sec:inclines}
The following test was conducted from a standing start and then walking up and down a hill or incline. The surface  of the incline was hard and somewhat uniform. There is much less focus on the heel part of the foot when walking up and down. It is also clear that much shorter and more frequent steps was taken when walking downhill. 

\begin{figure}[!htb]
    \centering
    \includegraphics[width = 1.1\linewidth]{uphill.png}
    \caption{Graph that shows the force applied to each FSR cell when walking up a hill/incline}
    \label{fig:uphill}
\end{figure}

\begin{figure}[!htb]
    \centering
    \includegraphics[width = 1.1\linewidth]{downhill.png}
    \caption{Graph that shows the force applied to each FSR cell when walking down a hill/incline}
    \label{fig:downhill}
\end{figure}

\clearpage
\section{Stairs}
The following test was conducted from a standing start and walking up and down stairs. The stairs had a small surface area; therefore, when walking on them, only a part of the foot would make contact. When walking up the stairs it is apparent that mainly the front area made contact with the ground. It is also interesting that the last step on the graph shows force applied to the heel area again, when the top of the staircase is reached.
\label{sec:stairs}
\begin{figure}[!htb]
    \centering
    \includegraphics[width = 1.1\linewidth]{upstairs.png}
    \caption{Graph that shows the force applied to each FSR cell when walking upstairs}
    \label{fig:upstairs}
\end{figure}
\begin{figure}[!htb]
    \centering
    \includegraphics[width = 1.1\linewidth]{downstairs.png}
    \caption{Graph that shows the force applied to each FSR cell when walking downstairs}
    \label{fig:downstairs}
\end{figure}




